\documentclass{article}
\usepackage{fullpage}
\usepackage{authblk}
\date{}
\title{Genetic architecture of developmental traits in populations of male gypsy moths}
\author[1]{Christopher J. Friedline}
\author[1,2]{Kristine Dattelbaum}
\author[1]{Erin M. Hobson}
\author[1]{Brandon M. Lind}
\author[1]{Rodney J. Dyer}
\author[3]{Dylan Parry}
\author[1]{Derek M. Johnson}
\author[1]{Andrew J. Eckert}
\affil[1]{Department of Biology, Virginia Commonwealth University}
\affil[2]{Deparment of Biology, University of Richmond}
\affil[3]{College of Environmental Science \& Forestry, State University of New 
York}
\begin{document}

\maketitle

\label{sec:abstract}

Local adaptation, characterisic of increased relative fitness
of a population of a particular genoytpe over genotypes originating from
other habitats, is an important consequence of natural selection
operating on fine spatial scales.  In this study, we examine the
pattern and process of local adaptation and genetic architecture of
developmental traits potentially influencing dispersal capacity
(larval mass, pupal duration, and developmental time) in male gypsy
moths (\textit{Lymantria dispar} L.), a particularly damaging,
non-native, invasive species.  A common garden was established in VA
from a local population in addition six other populations from
different regions of Virginia, North Carolina, and New York as well as
sites in Canada; a fully-factorial design is currently under
development.  In this talk, I present results from a single reference
assembly derived from a single individual, created from a paired-end
sequencing run on an Illumina HiSeq 2500 instrument.  To this
assembly, 192 barcoded individuals, from the common garden (differing
in source population) were mapped resulting in an individual-genotype
matrix containing thousands of reliable SNPs.  Bayesian Variable
Selection Regression (BVSR) as well as standard linear mixed-model
approaches were employed to explore the relationship between these
complex phenotypes and the underlying genetic variation across
populations. Understanding the genetic architecture of these
characters can serve to inform the control of gypsy moth spread as
well as increase understanding of the role of selective forces on
invasive species.


\end{document}
